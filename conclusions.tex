In this chapter, we summarize the main contributions of this dissertation,
and discuss potentially fruitful directions for future work.

\section{Summary}
As neural networks make their way into safety-critical systems, providing
formal guarantees about their safety has become paramount. This
dissertation focuses on \textit{differential} safety properties, such as
neural network equivalence, and it proposes \textit{accurate} and
\textit{general} algorithms for proving or disproving them. We proposed
four novel techniques towards this end.

In Chapter~\ref{ch:reludiff}, we proposed a novel technique for
over-approximating the difference between two structurally similar neural
networks. Our technique approximates the difference between the two
networks layer-by-layer, as opposed to existing techniques, which analyze
the two networks independently. To implement this technique, we formally
derived equations that relate the intermediate computations of the two
networks, and then used interval analysis to bound the difference. In
addition, we proposed an approach to iteratively refine the approximation.
We demonstrated that this new technique significantly improves both the
accuracy of the approximations and runtime to prove equivalence.

In Chapter~\ref{ch:neurodiff}, we extended the layer-by-layer approximation
technique with novel symbolic approximations. To implement this technique,
we proposed novel linear approximations for the difference between two
(ReLU) neurons, and an approach to introducing new symbolic variables to
represent the intermediate neurons of the two networks. We demonstrated
that this new symbolic technique further improves both the accuracy and
runtime for proving equivalence.

In Chapter~\ref{ch:onlinesyn}, we solved in a key challenge in making
differential verification general, by proposing a technique for
automatically synthesizing linear approximations for arbitrary functions.
To implement this technique, we designed an efficient synthesis algorithm
that combines inexpensive heuristic techniques with an SMT solver. We
demonstrated that our synthesized linear approximations are more accurate
than the best alternative approach that uses hand-crafted approximations.

In Chapter~\ref{ch:offlinesyn}, we proposed an offline technique for
synthesizing linear approximations to eliminate the overhead incurred by
the use of solvers in the online technique. This approach synthesizes a
static imperative program (that does not call solvers) that computes the
linear approximation, thus eliminating the overhead from expensive solvers.
We demonstrate that this approach again improves the accuracy over
hand-crafted approximations, though it trades accuracy for efficiency
compared to the online synthesis technique.

\section{Future Work}
Here we lay out potential extensions of our work.

\subsection{Safety Properties Beyond Equivalence}
\textit{Bounding Input Sensitivity.} Another important differential safety
property is to prove a bound on the sensitivity of a neural network's
inputs. Informally, this means proving that ``small changes to the inputs
only cause small changes to the outputs''. Formally, for a neural network $
f(x) $, an input space $ X $, a sensitivity bound $ \epsilon $, and a
perturbation $ r $, we would aim to prove $ |f(x) - f(x + r)| \leq \epsilon
$ for all $ x \in X $.

\textit{Hyperproperties.} Related to differential properties are
\textit{hyperproperties}. Here we give an example of a hyperproperty, which
we believe has not been considered by prior work in the context of neural
network verification. Informally, say we have a neural network that sensor
inputs from an autonomous vehicle, and outputs a steering direction. A
property we may be interested in proving is: ``given that in the previous
time-step, the neural network output a right turn, in the current
time-step, the neural network should continue to output a right turn or
continue straight''.


Formally, let $ f(x) $ be the neural network controller that outputs a
steering angle for an autonomous vehicle. Here, $ x \in \mathbb{R}^n $ is a
real-valued vector of sensor inputs, and assume $ f(x) \in [-30, 30] $ is
the rotation of the steering wheel, so that $ f(x) < 0 $ is a left turn, $
f(x) > 0 $ is a right turn, and $ f(x) = 0 $ goes straight. Furthermore,
let $ x_t $ be the input given to the neural network at time-step $ t $.
The property we wish to prove is that $ f(x_t) > 0 \implies f(x_{t+1}) \geq
0) $. Here, $ x_{t+1} $ would be defined by $ x_t $ and the system
dynamics. See e.g.~\cite{hu2020reach} for an example.

\subsection{Synthesizing Other Types of Approximations}
Many types approximations beyond linear approximation have been used in
neural network verification. Automatically synthesizing these
approximations would be interesting as well. We highlight some of these
below.

\textit{Zonotopes.} Zonotopes~\cite{ziegler2012lectures} have been used
extensively in neural network
verification~\cite{singh2018fast,bonaert2021fast,muller2021effective} are
very similar to linear approximations, with the main difference being that
the slopes of
the lower and upper bound must be the same. Zonotopes could be synthesized
both online and offline. For online synthesis, one would only need to
devise a suitable linear program for synthesizing the candidate zonotope,
as in Section~\ref{onlinesyn:sec:method-1}. Verifying that the zonotope is
a sound approximation is straightforward. After devising an online
approach, this procedure could then be used to generate examples for
offline synthesis. A key challenge for offline synthesis is that the
templates developed in Chapter~\ref{ch:offlinesyn} would likely not be
applicable to zonotopes. One would either need to train a neural network to
directly predict the zonotope's parameters, or devise new templates.

\textit{Multi-Neuron Approximations.} A recent fruitful direction for
improving accuracy of approximations in the context of neural network
verification has been \textit{multi-neuron}
approximations~\cite{Singh2019krelu,muller2021prima}. Whereas our online
and offline synthesis techniques consider only a single neuron in
isolation, prior work has shown computing approximations for multiple
neurons jointly can significantly improve accuracy. However, these
approximations are hand-crafted by experts, and thus could benefit from
synthesis.

\textit{Semi-Definite Relaxations.} Similar to linear approximations are
semi-definite relaxations, which permit \textit{quadratic} lower and upper
bounds. The added expressiveness of quadratic bounds versus linear bounds
has been shown to result in significantly more accurate
approximations~\cite{dathathri2020enabling}, though at the cost of
additional runtime because a quadratic program must be solved as opposed to
a linear program. Again, these techniques currently require and expert to
handcraft the approximation, and thus could benefit from synthesis.


\subsection{Improvements to Offline Synthesis}
As previously mentioned, the offline synthesis technique trades accuracy
for efficiency, however we believe we can achieve a better trade-off. In
Section~\ref{offlinesyn:sec:soundness}, we solve
Equation~\ref{offlinesyn:eq:opt} to compute the maximum violation, and
adjust all linear approximations by this amount. In many cases, we found
that this adjustment is relatively large, which means that often the
computed linear approximations will be looser than necessary.

A potential improvement would be to use \textit{counter-examples} to refine
the synthesized approximation when the maximum violation is above some
threshold. The underlying solver returns an input that maximizes
Equation~\ref{offlinesyn:eq:opt} (i.e. a counter-example), which provides a
hint at the inputs that are causing the violation to be large. We could
then take this counter-example and attempt to reduce the violation for this
input (and nearby inputs), either by retraining the neural network, or
re-partitioning the input interval space. This could be done repeatedly
until the maximum violation is reduced.