In this chapter, we review related work. We begin with the work that is most
closely related to the two key problems studied in our work: proving/disproving
differential properties, and synthesizing linear approximations. We then review
more general related work.

\section{Related Work on Differential Verification}
% Various heuristic based approaches, focus on discovering but not proving the
%absence of adversarial examples
Most closely related to analyzing differential properties of neural networks are
differential \textit{testing} techniques~\cite{xie2019diffchaser, PeiCYJ17,
MaLLZG18, asyrofi2021can}. These works focus on \textit{disproving} a differential
property, but, unlike our work, cannot prove that they hold.

We note that two recent works ~\cite{teuber2021geometric,kleine2020verifying}
propose techniques targeted proving equivalence of two \textit{arbitrary} neural
networks (whereas we target structurally similar networks). This work is
orthogonal to ours -- our technique significantly outperforms theirs when the
weights of the two DNN's are relatively close, whereas their technique does
significantly better when there is no relation between the two neural networks'
weights.

Also closely related are equivalence preserving neural network compression
techniques~\cite{lahav2021pruning,serra2021scaling}, also referred to as ``exact
neural network compression.'' Here, the equivalence verification is implicit (i.e.
there is no equivalence verification routine), however we believe inspiration for
future work could be drawn from these works.


\section{Related Work on Synthesizing Linear Approximations}
To the best of our knowledge, there is no work that directly addresses
automatically synthesizing linear approximations, however computing linear
approximations for arbitrary functions is a core component for rigorous global
optimization~\cite{chabert2009contractor} and SMT solving for nonlinear
theories~\cite{gao2013dreal}. These approaches work by \textit{decomposing} the
target function into its elementary operations, and combining linear
approximations for these elementary operations, to obtain a linear approximation
for the target function as a whole. While this produces a sound approximation, we
show in Chapters~\ref{ch:onlinesyn} and~\ref{ch:offlinesyn} that it is often
overly-conservative, which results in poor accuracy. A recent neural network
verification framework~\cite{autolipra} made it possible to easily apply this
decomposing-based approach, but supporting arbitrary activation functions was not
the original intent.

\section{Related Work on Verification of Neural Networks}
\textit{Approximation-based Neural Network Verification.}
There is a large body of work on using linear approximation
techniques~\cite{SinghGPV19,zhang2018efficient,shi2020robustness,boopathy2019cnn,WengZCSHDBD18,paulsen2020reludiff,paulsen2020neurodiff,wu2021tightening,mohammadinejad2020diffrnn}
and other abstract domains such as concrete intervals, symbolic
intervals~\cite{WangPWYJ18}, and Zonotopes~\cite{GehrMDTCV18},
for the purpose of neural network verification.
%
All of these can be thought of as leveraging restricted versions of the
polyhedral abstract domain~\cite{CousotH78,CousotC77}.
%
To the best
of our knowledge, these approaches are the most scalable (in terms of network
size) due to the use of approximations, but this also means they are less
accurate than exact approaches. In addition, all these approaches have the
limitation that they depend on bounds that are hand-crafted by an expert.
%
In addition, several works used semi-definite
relaxations~\cite{dathathri2020enabling,hu2020reach}, which essentially allow for
quadratic lower and upper bounds, as the approximations. These approximations
allow for more expressive lower and upper bounds, and thus often outperform linear
approximations in terms of accuracy. However, this approach adds a significant
amount of overhead when the lower and upper bounds need to be evaluated, thus they
are significantly less scalable.

\textit{SMT solver-based Neural Network Verification}
There is also a large body of work on using exact constraint solving for neural
network verification. Early works include solvers specifically designed for
neural networks, such as Reluplex and
Marabou~\cite{KatzBDJK17,KatzHIJLLSTWZDK19} and others~\cite{DvijothamSGMK18},
and leveraging existing
solvers~\cite{Ehlers17,HuangKWW17,BastaniILVNC16,HuangKWW17,baluta2019quantitative,tjeng2019evaluating,hu2020reach}.
While more accurate, the reliance on an SMT solver typically limits their
scalability. More recent work
often uses solvers to refine the bounds computed by linear
bounding~\cite{Singh2019krelu,SinghGPV19iclr,WangPWYJ18nips,tran2019star,tran2020verification}.
Since the solvers leveraged in these approaches usually involve linear
constraint solving techniques, they are usually only applicable to piece-wise
linear activation functions such as ReLU and Max/Min-pooling.

\section{Related Work on Testing Neural Networks}
Other works are more geared towards single network testing, and use
white-box testing techniques~\cite{ma2018deepgauge, xie2019deephunter,
	SunWRHKK18, TianPJR18, odena2018tensorfuzz}, such as neuron coverage
statistics, to assess how well a network has been tested, and also
report interesting behaviors. Both of these can be thought of as
adapting software engineering techniques to machine learning. In addition, many
works use machine learning techniques, such as
gradient optimization, to find interesting behaviors, such as
adversarial examples\cite{KurakinGB17a, madry2017towards, NguyenYC15,
	XuQE16, Moosavi-Dezfooli16}. These interesting behaviors can then be
used to retrain the network to improve
robustness~\cite{GoodfellowSS15, RaghunathanSL18}. Again, these
techniques do not provide guarantees, though we believe they could be
integrated into with our techniques to quickly disprove a property.